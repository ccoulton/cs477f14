\documentclass[[12pts]{report}

\usepackage{qtree}
\usepackage{listings}

\title{CS 477 Homework \\ Assignment 01}
\date{September, 2014}
\author{Charles Coulton}

\setlength{\topmargin}{-1cm}
\setlength{\oddsidemargin}{0in}
\setlength{\textwidth}{6.5in}
\setlength{\textheight}{8.3in}


%%Currently default settings for indentation and symbols.
%%Try these by uncommenting this block!!!
%%Redefine the first level symbols
%\renewcommand{\theenumi}{\fnsymbol{enumi}-}
%\renewcommand{\labelenumi}{\theenumi}
%
%%Redefine the second level symbols
%\renewcommand{\theenumii}{\alph{enumii})}
%\renewcommand{\labelenumii}{\theenumii}
%
%%Redefine the third level symbols
%\renewcommand{\theenumiii}{\roman{enumiii}.}
%\renewcommand{\labelenumiii}{\theenumiii}
%
%%Options for redefining levels


%\arabic
%\alph 
%\Alph
%\roman
%\Roman
%\fnsymbol
%This ^^^ is all you need to change!!

\begin{document}

\maketitle

\begin{enumerate}
	\item  
	Sort functions by growth rate.
	\begin{enumerate}
		\item
		$2^{\sqrt{log{n}}}$
		\item
		$2^{4/3}$
		\item
		${n}{{log}^3{n}}$
		\item
		$n^{log{n}}$
		\item
		$2^n$
		\item
		$2^{2^n} or 4^n$
		\item
		${2^{{n}^2}}$
	\end{enumerate}
	\item
	Prove functions using induction proofs prove the following:
	\begin{enumerate}
		\item
		Equation: $$\sum_{i=1}^n i(i)! = (n+1)! -1$$ \\
		Solve for Base step $n =1$
		$1!*1 = (1+1) -1$ leads to  $1*1 = 2 -1$ which equals $1 = 1$\\
		Next since that was true Assume for n+1 is true so induction step:\\
		$$\sum_{i=1}^{n+1} i(i)! = ((n+1)+1)!-1$$ \\
		In so that $\sum_{i=1}^n i(i)! + (n+1)*(n+1)!= \sum_{i=1}^{n+1}(i)*(i)!$ and since we found that\\ 
		$\sum_{i=1}^n i(i)! = (n+1)! -1$ in the last step lets replace these into the previous statements.\\
		$$(n+1)!-1 +(n+1)*(n+1)! = ((n+1)+1)!-1$$ \\
		With that cancel the -1 on both sides, and factor the (n+1)! from the items, which makes the equation look like this: \\
		$$(n+1)!(1+(n+1)) = (n+2)!$$ \\
		Given that n factorial is defined as n*(n-1)...*2*1, or n*(n-1)! so (n+2)! = n+2*(n+1)!.  With this in mind we can divide and cancel the (n+1)! on both sides, making the equation look thus: \\
		$$(n+1)!(n+2) = (n+2)(n+1)!$$
		This makes the equation balance at n+2 = n+2.\\
		\item
		Equation:
		$$\sum_{i=1}^n i^3 = |\frac{n(n+1)}{2}|^2$$
		Basis step n = 1, $1^3 = |\frac{1(1+1)}{2}|^2$ solving these equations you would get 1 = 1*4/4 so 1 = 1 \\\\
		Induction step: Assume that n+1 is true so that 
		$$\sum_{i=1}^{n+1} i^3=|\frac{(n+1)((n+1)+1)}{2}|^2$$\\
		With that in place we can do the replacements of $\sum_{i=1}^{n+1} i^3 = \sum_{i=1}^ni^3 + (n+1)^3$thus with that in place we can replace the other sum because of the last step stating $\sum_{i=1}^n i^3= |\frac{n(n+1)}{2}|^2$, so with those we can make the equation look like this:
		$$|\frac{n(n+1)}{2}|^2+(n+1)^3 = |\frac{(n+1)(n+2)}{2}|^2$$
		If we simplify the squares, and multiply the middle term by 4 and divide by 4 we get the equation like:
		$$\frac{n^2(n+1)^2}{4} + \frac{4(n+1)^3}{4} = \frac{(n+1)^2(n+2)^2}{4}$$
		Canceling the common 4 term under all the terms and factoring $(n+1)^2$ terms gets us to this:
		$$(n+1)^2*(n^2+4(n+1)) = (n+1)^2(n+2)^2$$
		Next step is to cancel the $(n+1)^2$ terms on both sides, after this Foil the $(n+2)^2$ term and distrubute the 4 inside the parathese giveing us finally:
		$$n^2+4n+4 = n^2+4n+4$$
	\end{enumerate}
	\item 
	Find the recurrance but don't solve it for:\lstset{language = C++, basicstyle =\footnotesize}
	\begin{lstlisting}
	int Example(int n)
		{
		if (n == 1)
			return;
		for(int i = 1; i<= n; i++)
			x += 1;
		y = example(n/2) + example(n-1);
		}
		
	  \end{lstlisting}
	
	
\end{enumerate}
\end{document}
