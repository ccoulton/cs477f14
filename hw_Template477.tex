\documentclass[[12pts]{report}

\usepackage{qtree}
\usepackage{listings}

\title{CS 477 Homework \\ Assignment 01}
\date{September, 2014}
\author{Charles Coulton}

\setlength{\topmargin}{-1cm}
\setlength{\oddsidemargin}{0in}
\setlength{\textwidth}{6.5in}
\setlength{\textheight}{8.3in}


%%Currently default settings for indentation and symbols.
%%Try these by uncommenting this block!!!
%%Redefine the first level symbols
%\renewcommand{\theenumi}{\fnsymbol{enumi}-}
%\renewcommand{\labelenumi}{\theenumi}
%
%%Redefine the second level symbols
%\renewcommand{\theenumii}{\alph{enumii})}
%\renewcommand{\labelenumii}{\theenumii}
%
%%Redefine the third level symbols
%\renewcommand{\theenumiii}{\roman{enumiii}.}
%\renewcommand{\labelenumiii}{\theenumiii}
%
%%Options for redefining levels


%\arabic
%\alph 
%\Alph
%\roman
%\Roman
%\fnsymbol
%This ^^^ is all you need to change!!

\begin{document}

\maketitle

\begin{enumerate}
	\item  
	Sort functions by growth rate.
	\begin{enumerate}
		\item
		
		\item 
		Formatting is all taken care of.
	\end{enumerate}
	\item
	And you can write after new items, \\
	With newline characters to seperate stuff...
	\item The last item! \\
	This has been nested enumeration!
\end{enumerate}

\noindent
Here is a tree that I made for CS326. \\
I didn't get full credit, so don't copy this! But this is the format of trees if you need to use them. \\

\Tree [.AssignStmt [.Var [.a[Index] [.a[E] [.a[2] ]]]] [.= ] [.E [.E [.Var [.b ]]] [.+ ] [.E [.1 ]]]] \\

Here is an equation: \\
$$\sum_{i=0}^{n} \frac{i(i+1)}{2} = n^{2} + 2n + 1$$

Here is the same equation $\sum_{i=0}^{n} \frac{i(i+1)}{2} = n^{2} + 2n + 1$ inline. \\

\noindent
Here is a c++ program: 
\lstset{language=C++,basicstyle=\footnotesize}
\begin{lstlisting}
#include<iostream>

using namespace std;

int main()
{
	cout<<"Hello World"<<endl;
	return 0;
}
\end{lstlisting}

\noindent
Here is a Python Program: 
\lstset{language=C++,basicstyle=\footnotesize}
\begin{lstlisting}
def f(x):
	return math.cos(x) - x
def fPrime(x):
	return -math.sin(x) - 1

def newtonRec(a,eps = 10**-5):
	A = a - f(a) / fPrime(a)
	if abs(A - a) < eps:
		return A
	else:
		return newtonRec(A,eps)
print newtonRec(.05)
\end{lstlisting}
\end{document}
